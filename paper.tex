% 如果在自己电脑编译, 编译顺序: pdflatex => bibtex => pdflatex => pdflatex, 在 Overleaf 则不用调整
% 主题分类调整请见 CombPaper.cls 文件
% 参考文献信息请加在 reference.bib 中, 然后在文中使用 \cite{} 进行引用
% 合作时可以使用 \mnote{} 命令进行批注

\documentclass{CombPaper}
\begin{document}
\title{\heiti 组合优化论文模板}

\author{作者一 \quad 作者二 \quad 作者三}
\address{上海财经大学数学学院, 上海 200433}
\keywords{组合数学}

\email{email1@sufe.edu.cn, email2@sufe.edu.cn, email3@sufe.edu.cn}

\maketitle

\begin{abstract}
	这里是摘要部分
\end{abstract}

\section{引言}
引言写作要求:
\begin{enumerate}
	\item 研究背景介绍:阅读主要参考文献\cite{gessel2005miki}以及:
	\begin{itemize}
		\item 阅读该文章中引用到的\cite{miki1978relation}和\cite{dilcher1996sums}, 并在引言中对文中的方法和主要结果做简要的介绍.
		\item 阅读引用到\cite{gessel2005miki}的文章如\cite{fu2009symmetric}和\cite{dilcher2016general}.当然, 也鼓励自行阅读其它文献.
	\end{itemize}
	\item 声明本文的主要创新点(可以以性质或定理的形式给出)
	\item 最后介绍本文的结构(可以梳理引理和定理)
\end{enumerate}

\section{正文}
基于\cite{gessel2005miki}, 创新点要求:
\begin{itemize}
	\item 最低要求: 大家需要比较第二类 Stirling 数作为 $m$ 的多项式两种表达式 $$S(m+n, m)=\beta_{n} m+\left(\beta_{n}^{(2)}+H_{n} \beta_{n}\right) m^{2}+\left(\beta_{n}^{(3)}+H_{n} \beta_{n}^{(2)}+H_{n, 2} \beta_{n}\right) m^{3}+\cdots$$ 和 $$S(m+n, n)=\beta_{n} m+\left(\frac{1}{2}(-1)^{n-1} B_{n-1}+\frac{1}{2} \sum_{i=1}^{n-1} \beta_{i} \beta_{n-i}\right) m^{2}+C m^{3}+\cdots$$ 中 $m^4, m^5$ 的系数, 给出等式.
	\item 进阶要求: 给出一般 $m^k$ 对应的两个表达式系数的等式.
	\item 开放性创新: 感兴趣的同学可以进行更深入的研究, 给出自己的创新性成果. 
\end{itemize}

正文可以按照论文实际情况拆分成不同章节. \mnote{批注: 合作时可以使用mnote命令批注}

\section{其它}
\textbf{总结及致谢部分非必须, 视个人情况而定.}

\bibliographystyle{abbrv}
\bibliography{./reference}

\end{document}